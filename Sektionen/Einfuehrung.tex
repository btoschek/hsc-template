\section{Einführung}
\label{sec:einfuehrung}

Bei diesem Dokument handelt es sich um eine \LaTeX{} Vorlage für wissenschaftliche Arbeiten in der \Acr{FEIF}. Es handelt sich hierbei nicht um eine Anleitung wie \LaTeX{} funktioniert, sondern rein um eine Vorlage die den formalen Richtlinien der \acr{FEIF} entspricht.

\newpage
\section{\LaTeX}

\LaTeX{} gehört zu den Typesetting-Sprachen und kann auf den ersten Blick ziemlich abschreckend wirken.
Einmal erlernt erlaubt es \LaTeX{} jedoch, Text schnell "auf Papier zu bringen". Dies liegt vor allem daran,
dass sich der Autor lediglich Gedanken um den Inhalt eines Dokuments machen muss, und das Layout sowie
bestimmte Verzeichnisse automatisch generiert werden. Nachfolgend wird ein kurzer Einblick in die Grundlagen
von \LaTeX{} gegeben.

\subsection{Inhalt}

Das Inhaltsverzeichnis wird durch die Verwendung von Überschriften im Text automatisch erstellt.
Wird eine neue Überschrift hinzugefügt, taucht diese an der entsprechenden Stelle im Inhaltsverzeichnis auf.
Wird eine Überschrift gelöscht, so wird sie auch aus dem Inhaltsverzeichnis entfernt. \\

Zudem gibt es noch weitere Verzeichnisse, die automatisch eingelayoutet werden.

\subsubsection{Abbildungen / Bilder}

Auch das Abbildungsverzeichnis wird automatisch erstellt, hierzu muss eine Abbildung mit
entsprechender Syntax eingebunden werden. In diesem Beispiel wurde in Abbildung
\ref{fig: Lenna} das bekannte Standardtestbild für Bildbearbeitung "Lenna" eingebunden.

\begin{figure}[H]
  \centering
  \includegraphics[width=.45\textwidth]{Lenna}
  \caption{Standard-Testbild für Bildbearbeitung "Lenna"}
  \label{fig: Lenna}
\end{figure}

Durch den "\textbackslash{}caption"-Befehl taucht die Grafik automatisch im Abbildungsverzeichnis auf.

\subsubsection{Tabellen}

Tabellen mit \LaTeX{} zu erstellen ist anfangs zugegebenermaßen etwas umständlich. Deswegen hier ein einfaches
Beispiel für eine Tabelle:

% Notiz: \hline erzeugt eine horizontale Linie

\begin{table}[H]
  \centering
  \begin{tabular}{|c|c|c|}
    \hline
    Spalte 1 & Spalte 2 & Spalte 3 \\
    \hline
    1.1 & 1.2 & 1.3 \\
    2.1 & 2.2 & 2.3 \\
    \hline
  \end{tabular}
  \caption{Beispieltabelle}
  \label{tab: Beispieltabelle}
\end{table}

Der "\textbackslash{}caption"-Befehl lässt die Tabelle wieder automatisch im Tabellenverzeichnis auftauchen.

\subsubsection{Codesnippets}

Codesnippets werden automatisch mit dem entsprechenden "Syntax Highlighting" angezeigt.
Hierbei lässt sich die Sprache pro Snippet ändern.
\vspace{.5cm}

\begin{lstlisting}[language=Python,caption=MD5-Hash-Generierung in Python]
import hashlib
password = '<PASSWORD>'
hashed = hashlib.md5((password + '5aM-2').encode()).hexdigest()
print(hashed)
\end{lstlisting}

\begin{lstlisting}[language=HTML,caption=HTML-Beispiel]
<body>
  <div class="example" id="test">
    <p>HTML Test</p>
  </div>
</body>
\end{lstlisting}

\subsection{Vorteile von \LaTeX}

\LaTeX{} macht es dem Autor besonders einfach, komplexe mathematische Formeln zu beschreiben, hier einige
Beispiele:

\begin{equation}
  \overline{x} = \frac{1}{n} \sum_{i=1}^n x_i
\end{equation}

\begin{align*}
  \int_0^2 x^2 &= 5 \\
  \lim_{x\to\infty} f(x) &= \sqrt{\ldots}
\end{align*}

Das Zitieren von verschiedenen Dokumenten ist in \LaTeX{} ebenso relativ einfach:

"Eine Faltung zweier Funktionen im Zeitbereich gestaltet sich als kompliziert, eine Vereinfachung
bringt es, die Funktionen in den Laplace-Bildbereich zu überführen.
Die Faltung im Zeitbereich entspricht einer Multiplikation im Bildbereich." \cite[S. 339f]{Papula2006}

Hierzu bietet es sich an, in einer genaueren Dokumentation zum Thema nachzulesen (Achtung: Englisch):
\url{https://www.overleaf.com/learn/latex/Bibliography_management_with_bibtex}

\section{Funktionalität}

Diese Vorlage wurde als Nachfolger der HS-internen Vorlage entworfen, welche nicht ohne Weiteres
erweiterbar war. Im Zuge der Umgestaltung wurden zahlreiche nicht benötigte Zeilen
gelöscht und eine für den Nutzer einfachere Umgebung geschaffen.

Verzeichnisse werden nun nur eingefügt, wenn diese auch benötigt werden (siehe \Gls{Glossar}). Sollten Packages fehlen oder
Einstellungen hinzugefügt werden wollen, so kann einfach eine Datei "Custom.tex" im gleichen Ordner wie
"Arbeit.tex" erzeugt werden. Diese Datei wird automatisch nach dem Laden der Vorlage ausgeführt
und besitzt somit die Möglichkeit, bereits gesetzte Werte und Variablen zu überschreiben.

Der Code wurde ausführlich kommentiert und sollte für Jeden ohne viel Mehraufwand veränderlich sein.

\subsection{Vorlagen-spezifische Funktionen}

Diese Vorlage stellt einige Funktionen zur Verfügung, die spezifisch für den Anwendungsfall sind.

\todo[inline,nolist,color=red!25,bordercolor=red]{!!! Die Definition des Templates liegt im Ordner "framework".
In diesem Ordner sollten keine Dateien gelöscht oder verändert werden.}

\subsubsection{Verwendung von Akronymen}

Bitte für Akronyme ausschließlich die unteren Kommandos benutzen, da das Abkürzungsverzeichnis
sonst nicht angezeigt wird.

\begin{itemize}
  \item \textbackslash{}acr\{xyz\} für Kurzform des Akronyms
  \item \textbackslash{}Acr\{xyz\} für die lange Form des Akronyms
\end{itemize}

\subsubsection{Symbolverzeichnis}

Über "\textbackslash{}nomen\{<ZEICHEN>\}\{<BESCHREIBUNG>\}" kann ein neuer Eintrag im
Symbolverzeichnis erzeugt werden. Die nachfolgenden Kommandos erleichtern die Erstellung
eines Eintrags, sind jedoch nicht zwangsläufig notwendig.

\begin{itemize}
  \item \textbackslash{}nomunit\{xyz\} für die rechte Spalte
  \item \textbackslash{}nomsi\{\textbackslash{}metre\} für die formatierte Einheit (hier Meter)
\end{itemize}

\subsubsection{Fortschrittsmanagement}

Wenn man an einem stetig wachsenden Dokument arbeitet, kann es sein,
dass einem einfällt, etwas an anderer Stelle im Text zu ändern oder zu überarbeiten.
Um nicht alles separat oder sogar mit Block und Stift aufzuschreiben (wofür benutzt man schließlich
einen Computer), stellt die Vorlage einige Kommandos zur Verfügung, um Anmerkungen
automatisch einzulayouten.

\begin{itemize}
  \item \note{Beispielnotiz}\textbackslash{}note\{Beispielnotiz\} für eine neue Notiz am Rand
  \item \unsure{Überarbeitung anstehend}\textbackslash{}unsure\{Überarbeitung anstehend\}: eventuelle Überarbeitung später
  \item \change{Änderung}\textbackslash{}change\{Änderung\}: Änderungsidee
\end{itemize}

Sobald TODOs im Dokument vorhanden sind, werden diese auf der letzten Seite aufgeführt.

\subsubsection{Tabellen mit Kopfzeile}

Um Tabellen mit Kopfzeile einzufügen, kann die "\textbackslash{}colortable"-Umgebung genutzt werden.

\begin{table}[H]
  \centering

  % Custom tabular environment
  \begin{colortable}{|c|c|c|}
    Spalte 1 & Spalte 2 & Spalte 3 \\
    \tablecontent
    1.1 & 1.2 & 1.3 \\
    2.1 & 2.2 & 2.3 \\
  \end{colortable}

  \caption{Custom Tabelle}
  \label{tab: Tabelle 2}
\end{table}

\subsubsection{Referenzen}

Um Referenzen auf Abbildungen syntaktisch zu verkürzen, stellt das Template
ein eigenes Makro zur Verfügung: "\textbackslash{}imgref\{<LABEL>\}". Gleiches gilt
für Tabellen ("\textbackslash{}tabref\{<LABEL>\}").

Referenz auf Abbildung \ref{fig: Lenna}: \imgref{fig: Lenna}\\
Referenz auf Tabelle \ref{tab: Tabelle 2}: \tabref{tab: Tabelle 2}